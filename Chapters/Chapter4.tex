\chapter{Experiments}
In this section, I will discuss how I evaluate the LSH with some real world data, and comparison with other spatial indexes. Finally, present the outcome that is measured from various aspects.

\section{Experimental Setup}

For evaluation, the LSH is implemented in C++ and has been tested on a machine with 2.2GHz 6-core Intel Core i7 CPU and 16GB memory. Three sets of two-dimensional dataset were used: \texttt{post} \cite{rtreeportal}, \texttt{osm} \cite{OpenStreetMap}, \texttt{g2-2-10} \cite{G2sets}. 

\begin{center}
\begin{adjustbox}{max width={\textwidth}, max totalheight={\textheight},keepaspectratio}
\begin{threeparttable}
\caption{Datasets}

\begin{tabular}{c|c c c}
    \toprule
                        % & \multicolumn{3}{c}{Dataset} \\\midrule \midrule
    \textbf{Name}    &\textbf{Size}  & \textbf{No. of data pairs} & \textbf{Data type}             \\ \midrule 
    \texttt{post}    & 2.2M & 123,593             &Geographic Coordinates \\
    \texttt{osm}     & 3.0G & 114,932,854         &Geographic Coordinates \\
    \texttt{g2-2-10} & 18K  & 2,048               & Synthetic Clusters    \\ \bottomrule
\end{tabular}

% \begin{tablenotes}
% \item[1] qwerty; \item[2] asdfgh
% \end{tablenotes}
\end{threeparttable}
\label{table:datasets}
\end{adjustbox}
\end{center}

\textbf{Datasets}: As shown in the Table \ref{table:datasets} above, we have two real world datasets the \texttt{post} and \texttt{osm}, and one synthetic dataset the \texttt{g2-2-10}. The \texttt{post} dataset contains 123,593 coordinates of post offices in North America. The second dataset \texttt{osm} contains 114,932,854 coordinates in Australia and Oceania from OpenStreetMap. The \texttt{g2-2-10} is a synthetic cluster dataset that contains 2,048 points that are computed from Gaussian clusters with 2 dimensions and the standard deviation set to 10. 



\textbf{Competitors}:

\textit{R-Tree}


\textit{Learned ZM index}
