\chapter{Introduction}

Large volumes of data have been collected and generated considerably over the recent years, among the increasing popularity of smartphones and Internet services. Multidimensional data such as geographic coordinates data and CAD data is considered as core infrastructure and component in the technology world \cite{gunther1990research, morton1966computer}. Data queries in spatial data, such as range query and k nearest neighbour query has been widely used in map based products and location based services. Google maps use range queries and KNN to find points of interest. For example, finding restaurants in a region and finding restaurants near me. In VLSI CAD, spatial data index is also used to manage two-dimensional data objects such as polylines and polygons \cite{liu1994evaluation}. In some other areas, the spatial management system is also integrated with image processing and computer vision. For example, medical images and remote sensing images use spatial databases to store and location spatial objects from the images \cite{borah2004improved, adhikary1996knowledge, mantel2004matching, tagare1997medical}. 

How to effectively manage spatial data have been studied significantly in the past few decades \cite{Gaede:1998fp, ooi1990efficient}. Spatial index methods have been developed in many different approaches. Some have derived from a one-dimensional approach, such as linear hashing \cite{larson1980linear} and extendible hashing \cite{fagin1979extendible}. Some have derived from popular one-dimensional B-Tree structure \cite{Bayer:2002ds}, such as K-D-Tree \cite{Bentley:1975gn}, Quad-Tree \cite{CSUR:tm}, R-Tree \cite{Guttman:1984ka}, and so on. 


\section{Motivation}

\section{Research Questions}

\section{Approach and Outcomes}

\section{Thesis Structure}